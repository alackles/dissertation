Evolution sheds light on all of biology, and evolutionary dynamics underlie some of the most pressing issues we face today.  
If we can deepen our understanding of evolution, we can better respond to these various challenges.
However, studying such processes directly can be difficult; biological data is naturally messy, easily confounded, and often limited. 
Fortunately, we can use computational modeling to help simplify and systematically untangle complex evolutionary processes.
The aim of this dissertation is therefore to develop innovative computational frameworks to describe, quantify, and build intuition about evolutionary phenomena, with a focus on connectivity within evolvable systems. 
Here I introduce three such computational frameworks which address the importance of connectivity in systems across scales.

First, I introduce rank epistasis, a model of epistasis that does not rely on baseline assumptions of genetic interactions. Rank epistasis borrows rank-based comparison testing from parametric statistics to quantify mutational landscapes around a target locus and identify how much that landscape is perturbed by mutation at that locus. This model is able to correctly identify lack of epistasis where existing models fail, thereby providing better insight into connectivity at the genome level.

Next, I describe the comparative hybrid method, an approach to piecewise study of complex phenotypes. This model creates hybridized structures of well-known cognitive substrates in order to address what facilitates the evolution of learning. The comparative hybrid model allowed us to identify both connectivity and discretization as important components to the evolution of cognition, as well as demonstrate how both these components interact in different cognitive structures. This approach highlights the importance of recognizing connected components at the level of the phenotype.

Finally, I provide an engineering point of view for Tessevolve, a virtual reality enabled system for viewing fitness landscapes in multiple dimensions. While traditional methods have only allowed for 2D visualization, Tessevolve allows the user to view fitness landscapes scaled across 2D, 3D, and 4D. Visualizing these landscapes in multiple dimensions in an intuitive VR-based system allowed us to identify how landscape traversal changes as dimensions increase, demonstrating the way that connections between points across fitness landscapes are affected by dimensionality. 

As a whole, this dissertation looks at connectivity in computational structures across a broad range of biological scales. These methods and metrics therefore expand our computational toolkit for studying evolution in multiple systems of interest: genotypic, phenotypic, and at the whole landscape level. 