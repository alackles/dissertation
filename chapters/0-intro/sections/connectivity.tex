\section{Connectivity in Evolvable Systems}

\textit{Connectivity} is a term I use throughout this dissertation to refer broadly to the fact that in any complex evolvable system, some components of the system impose constraints on other components. Connectivity between genes creates epistasis (Chapter 2); connectivity between information structures creates cognition (Chapter 3); connectivity between mutational optima creates valley-crossing (Chapter 4). It is this connectivity which creates evolutionary phenomena worthy of deeper study--after all, an unconstrained system is an unlimited problem solver. 

There are numerous existing approaches to studying connectivity in evolvable systems. One approach is to look at evolutionary robustness, or how the failure of one component affects the performance of the system \citep{wagner_robustness_2007, masel_robustness_2009}. Robustness is frequently studied as a driver of genetic variation, particularly in metabolic networks \citep{handorf_expanding_2005, felix_robustness_2008}. A second common approach is modularity: the study of systems consisting of densely connected hubs which are sparsely connected to each other \citep{wagner_road_2007, clune_evolutionary_2013, melo_modularity_2016}. Modularity in evolution is a key driver of evolution at all scales, from
protein interaction networks \citep{han_evidence_2004} to
morphological innovation \citep{parsons_constraint_2012}
eco-evolutionary dynamics \citep{fletcher_network_2013}. In both these cases, however, connectivity is under study as an aspect of a more specific structural phenomenon.

This dissertation takes a very broad view of connectivity which draws from each of the areas listed above. I am interested in studying connectivity not as a feature of any particular system or phenomenon, but as an underlying factor of what makes evolvable systems evolvable. 

One challenge of studying connectivity in evolution is that natural systems are subject not only to the inherent constraints of connectivity itself but also to the historical constraints of past connections \citep{blount_historical_2008}. Therefore, to study connected systems at a theoretical level, it is necessary to study evolution not just as it has happened in existing systems, but as it happens in controllable systems: evolution in action.