\section{Digital Evolution and Artificial Life}

Biological data is naturally messy, easily confounded, and often limited. 
While I am interested in studying questions of connectivity and complexity, often natural systems introduce \textit{irrelevant} complexity, related more to the vagaries of a particular system than the underlying theory driving the system.
Fortunately, we can use computational modeling to help simplify and systematically untangle complex evolutionary processes.

Using computational models, we can study evolution as an algorithm. 
The mathematical rules which govern evolution in biological systems can be directly translated to computational systems.  
We can represent organisms as data structures, and evolutionary forces as an algorithmic process to evaluate, mutate, and replicate those structures in a process termed \textit{digital evolution} or \textit{artificial life}. 
I will use here the term artificial life as it is the broader term, but both are applicable.

Artificial life is a rapidly-growing and relatively recent field; for some reviews of artificial life as a field and its current applications, see 
\citep{oneill_digital_2003, mcmullin_thirty_2004, kim_comprehensive_2006, aguilar_past_2014}. 
The field has produced insights into eco-evolutionary processes such as
adaptation \citep{wilke_evolution_2001}, 
genetic complexity \citep{ostman_impact_2011}, 
sexual selection \citep{bohm_sexual_2020},
cooperation \citep{vostinar_spatial_2019}, 
and more.
At the cornerstone of many of these approaches is the desire to \textit{replicate} and therefore \textit{understand} biological evolution by means of digital evolution. 

Two decades, ago major scholars in the field identified fourteen key open problems in artificial life to help advance the field \citep{bedau_open_2000}. 
At least ten (numbers 2, 4, 5, 6, 7, 8, 9, 10, 11, and 13) deal with creating and analyzing novel artificial life representations. 
All fourteen open problems remain unsolved.

Solving these problems is not merely a question of conducting the right experiments or asking the right questions; in many cases, they require the construction of entire new systems to analyze complexities currently out of reach for computational systems. 
Incredible progress has been made towards these new systems, including 
the construction of an artificial cell \citep{frischmon_build--cell_2021}, 
the development of self-organizing bio-robots \citep{kriegman_scalable_2020}, and 
steps towards indefinitely scalable architecture \citep{ackley_indefinitely_2014}. 
Each of these systems contains some model of adaptive biology or evolution; it is that general-purpose evolutionary model that makes them powerful tools for problem solving.

Unsurprisingly, solving open problems about the nature of life itself is outside the scope of a single dissertation. 
However, what is clear is that creating informative models of evolving systems is an open problem of great interest to the fields of digital evolution and artificial life. 
I therefore introduce in this dissertation three new models based in digital evolution systems which seek to expand our repertoire of explanatory and exploratory models of evolution.