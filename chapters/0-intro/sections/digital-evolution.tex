\section{Digital Evolution and Artificial Life}

While I am interested in studying questions of connectivity and complexity, often in natural systems it is difficult to separate the complexity you are interested in from the complexity that arise from stochastic processes outside experimental control.
Fortunately, we can use computational modeling to help simplify and systematically untangle complex evolutionary processes.

Using computational models, we can study evolution as an algorithm. 
The mathematical rules which govern evolution in biological systems can be directly translated to computational systems.  
We can represent organisms as data structures, and evolutionary forces as an algorithmic process to evaluate, mutate, and replicate those structures in a process termed \textit{digital evolution} or \textit{artificial life} \cite{oneill_digital_2003}. 
I will use here the term artificial life as it is the broader term, but both are applicable.

Artificial life is a rapidly-growing and relatively recent field; for some reviews of artificial life as a field and its current applications, see 
\citep{oneill_digital_2003, mcmullin_thirty_2004, kim_comprehensive_2006, aguilar_past_2014}. 
The field has produced insights into eco-evolutionary processes such as
adaptation \citep{wilke_evolution_2001}, 
genetic complexity \citep{ostman_impact_2011}, 
sexual selection \citep{bohm_sexual_2020},
cooperation \citep{vostinar_spatial_2019}, 
and more.
At the cornerstone of many of these approaches is the desire to model and understand biological evolution by means of digital evolution.

Creating informative models of evolving systems is an open problem of great interest to the fields of digital evolution and artificial life.
Incredible progress has been made towards these models, including for example
the construction of an artificial cell \citep{frischmon_build--cell_2021}, 
the development of self-organizing bio-robots \citep{kriegman_scalable_2020}, and 
steps towards indefinitely scalable architecture \citep{ackley_indefinitely_2014}. 
Each of these systems contains some model of adaptive biology or evolution; it is that general-purpose evolutionary model that makes them powerful tools for problem solving. 
Crucially, while these models operate at very different scales from cells to systems, each is made of connected components which together form a complex network of interactions.
It is this complex network which allows evolutionary systems to operate as more than the sum of their parts, whether those parts are artificial organelles or large networks of computer hardware. 
Connectivity in digital systems is therefore an important component to creating rich, informative models of evolutionary processes.


I introduce in this dissertation three connectivity-focused approaches to studying digital evolution which seek to expand our repertoire of explanatory and exploratory models of evolution. Each of these models operates at different scales, but each address problems of connectivity in their respective evolvable systems.  