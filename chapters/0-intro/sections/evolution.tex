\section{Why Study Evolvable Systems?}

This dissertation approaches the study of evolution in a broad, theoretical sense rather than in a specific model system (hence the term \textit{evolvable system\textbf{s}}). The goal of this approach is to study evolution not in any particular instance, but to study it as a process that underlies a wide variety of present-day scientific challenges. Evolutionary processes govern, for example, ecological systems' responses to human activity \citep{nadeau_ecoevolution_2019},  antibiotic resistance and vaccine efficacy \citep{davies_origins_2010, kennedy_why_2018}, and the emergence of viral variants \citep{rochman_ongoing_2021, cobey_concerns_2021}. If we can deepen our understanding of evolution in a general sense, we can better respond to these various challenges. In particular, I focus here on what I argue is one of the key drivers of complex evolutionary processes: connectivity. 