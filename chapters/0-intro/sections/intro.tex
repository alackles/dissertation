Evolutionary theory is the through line of biology. At all scales from base pairs to ecosystems, evolution is a powerful driving force and organizing principle for biological phenomena. Understanding evolutionary processes is essential to understand the world in which we live and to address large-scale biological challenges such as disease prevention and climate change mitigation.

Evolution can be challenging to study, however, in part because evolvable systems\footnote{\textit{Evolvable systems }here refers to any system (digital, biological, theoretical) that adheres to and is shaped by principles of evolution: mutability, heritability, and selection.} are made of constrained, connected components. Those connections are what make evolved systems more than just simplistic input-output machines. Studying these connections in actively evolving systems allows us to make inferences about the fundamental properties behind evolutionary trajectories.

Since natural systems are subject to numerous confounding factors and stochastic processes, it is often more tractable to study evolution via a simulation or model. However, no single model can capture all the complexities of any system; if it could, it would be a direct copy providing no further insight. Imagine, for example, a map with perfect resolution of every street, every bend in the road, every pothole or sidewalk crack, every blade of grass. Such a map would be, by necessity, the size of the world. Therefore, simplifications must be made: streets as smooth lines, buildings as vague shapes. Similarly, to untangle evolutionary processes, we must approach them from a variety of angles and with a suite of imperfect yet informative models. 

The aim of this dissertation is therefore to develop innovative computational frameworks to describe, quantify, and build intuition about evolutionary phenomena, with a focus on connectivity within evolvable systems. Here I introduce the theoretical backing and methodological process for three new analysis techniques dealing with such evolutionary phenomena at the genotype, phenotype, and landscape level. These techniques are: a metric for epistatic interaction; an approach to the study of cognitive structures; and a visualization platform for multidimensional fitness landscapes. Together, these approaches represent a novel contribution to to the study of connected evolvable systems at a variety of scales.
