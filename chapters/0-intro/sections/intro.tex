Evolutionary theory is the through line of biology. At all scales from base pairs to ecosystems, evolution is a powerful driving force and explanatory factor for biological phenomena. Understanding evolutionary processes is essential to understand the world in which we live and to address the challenges we face.

One complexifying aspect of evolution is that evolvable systems are made of constrained, connected components. Those connections are what make evolved systems more than just simplistic input-output machines. Studying these connections in actively evolving systems allows us to make inferences about the fundamental properties behind evolutionary trajectories.

Since natural systems are subject to numerous confounding factors and stochastic processes, it is often more tractable to study evolution in some model system. However, no single model can capture all the complexities of any system; if it could, it would be a direct copy providing no further insight. Therefore, to untangle evolutionary processes, we must approach them from a variety of angles and with a suite of imperfect yet informative models. 

The aim of this dissertation is therefore to develop innovative computational frameworks to describe, quantify, and build intuition about evolutionary phenomena, with a focus on connectivity within evolvable systems. Here I introduce the theoretical backing and methodological process for three new models and metrics dealing with such evolutionary phenomena at a variety of scales.
