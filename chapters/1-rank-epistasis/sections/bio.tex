\subsection{Biological Data: Mutational Landscapes}


We tested the efficacy of our model on empirical data, where epistatic degree is not known \textit{a priori}. To make an accurate comparison we require biological data which is openly available, consisted of both single and pairwise mutations, and had been evaluated for epistatic interaction in the original study. Based on these criteria, we selected a dataset generated from experiments in the Long-Term Evolution Experiment (LTEE) conducted on \textit{Escheria coli} \citep{khan_negative_2011}. 

%We tested the efficacy of our model on empirical data, where epistatic degree is not known \textit{a priori}. We conducted a literature search for biological data which was openly available, consisted of both single and pairwise mutations, and had been evaluated for epistatic interaction in the original study. 
%We selected two datasets based on these criteria, one in \textit{Escheria coli} \citep{khan_negative_2011} and one in \textit{Saccharomyces cerevisiae} \citep{bank_predictability_2016}. 


\subsection{Performance on Biological Mutational Landscapes}

Discuss performance on E. coli landscape -- tracks the traditional metric pretty well I hope as the background changes. Will have to conduct further computation on the later aspects.

Need to conduct some of the analysis on the yeast landscape even though that one is way harder to wrangle the data for; you can work that out I hope. 


\subsection{Comparison to Metrics $\beta$ and $\gamma$}

\begin{figure}
    \centering
    \includegraphics[width=\textwidth]{figs/summary_bio_re.pdf}
    \caption{Caption}
    \label{fig:my_label}
\end{figure}


\begin{figure}
    \centering
    \includegraphics[width=\textwidth]{figs/summary_bio_trad.pdf}
    \caption{Caption}
    \label{fig:my_label}
\end{figure}

Hopefully the e. coli data will track the $\beta$ metric pretty well as time goes on.