\section{Conclusion}

Rank epistasis is a promising new metric to detect interactivity at specific sites in a genome without the need for a baseline assumption of additivity or multiplicativity; therefore, unlike existing metrics, there is no risk of selecting the wrong baseline assumption and falsely detecting epistatic activity where there is none. Additionally, rank epistasis is able to accurately detect a lack of epistasis when both additive and multiplicative baseline interactions are present, which existing metrics are unable to account for. The metric is also able to reveal areas of high mutational robustness within a genome.

One major difference between rank epistasis and existing metrics is that it does not give an indication of \textit{positive} or \textit{negative} epistasis, only the \textit{degree} of epistasis and how much perturbation is present in the ordering of possible mutants. Other metrics may be more appropriate if the sign of epistasis is of particular interest; however, if detection of site interaction and the overall density of the mutational landscape is the primary goal, rank epistasis can be a strong choice of metric. Additionally, it can be used to supplement sign epistasis metrics to ensure the measured epistasis is not an artefact of baseline assumptions.

 Here we do not provide biological applications for rank epistasis as it is beyond the scope of this paper. However, this metric could be easily applied to any sufficiently complete biological dataset in the same way it is applied here to computational ones. It is our hope that using rank epistasis in this way to detect genome connectivity in biological systems will lead to a fuller understanding of the complex ways in which genome structure dictates function.