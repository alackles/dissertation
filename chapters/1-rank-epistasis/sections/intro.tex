\section{Introduction}

Epistasis, or the interaction between two or more genes to create an unexpected phenotype, is a genetic phenomenon that creates much of the complex structure we see in genome evolution. Often, we are interested in epistasis not just for its own benefit, but rather as an explanation for how a population might shift from one stable genotype to another stable genotype. Both increased theoretical understanding \citep{weinreich_should_2013, payne_causes_2019} and long-term empirical studies \citep{khan_negative_2011, wiser_long-term_2013, gupta_shared_2016}, have led us to greater understanding of the significant role epistasis plays in evolution and adapation.

One major challenge to studying epistatic interactions, however, is \textit{detecting} epistatic interactions. While the impact of epistasis on genome function is well-established, there is still ongoing uncertainty as to how to measure such impact \citep{cordell_epistasis_2002,de_visser_causes_2011, mackay_epistasis_2014, niel_survey_2015}. Determining how strong epistatic interactions are at individual sites in a genome--or, indeed, whether sites are epistatic at all--is essential to understanding evolutionary landscapes \citep{weinreich_should_2013}. However, such detection can be confounded if we do not have an accurate baseline expectation for how genes ``should" interact in the absence of epistatic activity. 

Current measures of epistasis tend to set the assumption that genes will interact either additivly \citep{ostman_impact_2011} or multiplicativly \citep{elena_test_1997}, and measure epistasis based on the deviation from this proposed interaction. However, these assumptions each result in qualitatively different kinds of interactions being detected as epistatic \citep{puniyani_meaning_2004}. Therefore, if the wrong baseline is chosen, or if some loci interact additively and others multiplicatively, loci may appear epistatic upon measurement despite having no true epistatic effect.

In order to address these challenges, we propose a rank-based epistasis metric that avoids assumptions about the nature of epistatic interactions, focusing instead on how perturbations to one site affect the ordering of fitness contributions from other sites. We find that this approach correctly identifies \textit{lack of} epistasis in an example case where both additive and multiplicative existing metrics fail. Further, it is able to identify different degrees of interactivity within genomes without making assumptions about the nature of these interactions.