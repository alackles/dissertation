\section{Conclusion}
% FINAL?

We present these results as an illustrative example of the Comparative Hybrid Method and how it can be applied to analyze key components of digital brain architectures. 
By creating hybrids of existing examples of neuro architectures, we are able to test computational structures in a piece wise manner on a collection of tasks of varying complexity. 
The comparative performance of the unmodified and hybrid brains can be used to make inferences about how computational components and structure may relate to task success and ultimately to understanding fundamentally why the unmodified brains behave differently.

In practice, this technique can be extended to any two or more structures whose performance you are interested in comparing.
All that is required is an intuition for what architectural differences separate the two computational structures, and an implementation of hybridized versions altering each of those components in turn. 
While such a task may be in some cases practically unwieldy, it will in many cases be simpler than the alternative of analytically examining the entire brain architecture at once and altering all possible parameters in order to identify key components.

In theory, the effects of the comparative hybrid model could extend into biological systems. 
As the ability to accurately simulate neuron models progresses, our method could be used across a continuum of neuron models and with the right advances in bio-engineering, the biological brain itself. 
Ultimately we believe that the Comparative Hybrid Method will allow us to single out key variables of cognition that could provide a path to a better understanding of intelligence, whether it be biological or computational. 