\section{Conclusion}
Here we introduced Tessevolve, a platform for visualizing fitness landscapes in 2D, 3D, and 4D. The concept was inspired by a lack of existing tools for qualitatively understanding how fitness landscapes change as additional dimensions are added. It was designed to be intuitive, customizable, and extensible. Based on expert feedback, we conclude that Tessevolve is an effective and easy-to-use tool for gaining a stronger intuitive understanding of complex fitness landscapes.

Experts found that when viewing 3D and 4D landscapes, Tessevolve was generally easier to use and interpret than existing platforms they are familiar with. They also reported that the landscape visualization was easy to interpret, even when given minimal information beyond the built-in interface. We incorporated their feedback about adding a clearer and more apparent key to the visualization components to improve Tessevolve's usability and overall design. 

In the future, Tessevolve could incorporate additional sources of information for additional dimensionality. In particular, leveraging the audio input capabilities of VR headsets could allow us to expand into more dimensions, or replace color as a fitness indicator to make the visualization more accessible to blind or low-vision researchers.

Another improvement we are considering, which may be particularly helpful for large and complex landscapes, would be to add a mode inspired by the holey fitness landscape concept (see Domain Background). In this mode, the user would enter a fitness threshold that they were interested in viewing. Tessevolve would then render 3D shapes representing the regions of the search space with fitnesses below that threshold. This approach would allow the user to quickly build intuition about a larger region of the search space, at the cost of reducing the resolution at which fitness is displayed.

The generally positive feedback from domain experts indicates that Tessevolve is a valuable new tool for evolutionary biology and computational evolution. While the number of dimensions Tessevolve can display is still limited, the ability to compare properties across dimensions allows for insight into how dimensionality affects evolutionary dynamics. In particular, Tessevolve reveals the ways landscape ruggedness and landscape traversal intersect with the number of landscape dimensions. It is our hope that Tessevolve, and other VR-based visualizations, can continue to push the boundaries of human ability to comprehend complex fitness landscapes. It has been decades since the problems with building intuition from low-dimensional visual fitness landscape metaphors were first pointed out. Perhaps now we can finally begin building a better visual metaphor.