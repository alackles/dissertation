\section{Expert Feedback}
We solicited feedback from three domain experts whose primary disciplines were in three different fields in which evolutionary fitness landscapes are common: one in evolutionary and theoretical biology, one in machine learning and evolutionary computation, and one in digital evolution and artificial life. They were asked to rate the ability to use and interpret data in Tessevolve as easier, harder, or about the same as current methods they are familiar with, and then to provide open-ended responses about their experience with the platform. Expert feedback was generally positive, especially regarding interpretability across dimensions and the intuitive use of the visualization tool.  

\subsection{Ease of Interpretation}

Two of the three experts rated Tessevolve as \textit{easier to interpret} than current methods for both 3D and 4D data, with one pointing out that so few methods even exist for 3D and 4D data that they currently avoid attempting to visualize such data. These experts also provided qualitative feedback about the ease of interpretation:

\begin{displayquote}

\textbf{Expert 1:}  I had a hard time answering some of the questions [about comparative ease of use] since I don't generally try to visualize 3D or 4D fitness landscapes since I don't have tools to do so easily.

\textbf{Expert 2:} Overall I was pleasantly surprised by how easy it was to view both landscapes and lineages in 2D and 3D! I believe the 4D mode would become more useful as time goes on and you are better able to wrap your brain around it. That said, even with just a few moments I was able to see the effect of the fourth dimension on some landscapes, so it may not take long to build that intuition. 


\end{displayquote}


One expert found the 3D data harder to interpret than current methods, and was unsure about the interpretability of 4D data. However, their feedback on the tool indicated they were unable to find the explanation of positionality and color mappings provided on the website:

\begin{displayquote}

\textbf{Expert 3:} Moreover, I think you might add a line or two with the syntax of the visualisation; I am not a big expert in landscape visualisation, but I found it non-intuitive.

\end{displayquote}

This syntax was at the time provided in the FAQ, but based on this feedback we moved it to the front matter of the website.

All three experts found 2D data harder to interpret or about the same as current methods. This makes sense, as current methods such as topographical fitness landscapes are explicitly designed to handle 3D data, while Tessevolve explicitly chooses to sacrifice 2D interpretability in favor of improved ease of comparison to multiple dimensions. 

\subsection{Ease of Use}

Two of the experts agreed that the controls for Tessevolve were intuitive and that it ran well in the browser and in the Oculus headset:

\begin{displayquote}

\textbf{Expert 1:} I'm very impressed how well it runs in the browser and the controls were intuitive. 

\textbf{Expert 2:} I could definitely use this in my own work, as visualizing lineages in 3D is no easy task, even though it can really help build your understanding of the system. The VR support in particular is really useful for actually understanding the space in 3D.

\end{displayquote}

Expert 3 had some trouble rendering the 2D landscapes, but we and the other experts were unable to replicate this difficulty; they did not provide other comments related to ease of use.

\subsection{Constructive Feedback}

All three experts provided some constructive feedback on the visualization that we incorporated into our final design. 

As noted, one expert had trouble interpreting the landscape's design, and we attribute this to the fact that this information was in a pop-up menu in the FAQ. We therefore moved this information to the front of the page. We similarly moved information about the color scheme to the front page based on feedback from Expert 2 regarding uncertainty about whether the scheme transferred to the lineage data:

\begin{displayquote}

\textbf{Expert 2:} One thing I would suggest would be a guide to the color system for both the landscape and the lineage. It appears that the landscape shifts from purple to yellow as fitness increases, but the lineage colors are more difficult to deduce. 

\end{displayquote}

Given that experts generally found Tessevolve easier to use for multidimensional data, and critical comments were related to the front-facing interface rather than the visualization itself, this initial round of expert feedback indicates Tessevolve can be an effective tool for 3D and 4D landscape data visualization.