% some great advice from betty cheng
% https://www.cse.msu.edu/~chengb/Writing/intro-guidelines-stirewalt.txt
\section{Introduction}

Evolution, once firmly in the domain of biologists, is now understood to be a fundamental algorithmic process with wide applicability across fields. 
Computer scientists and engineers in particular have recently made many advances towards understanding and harnessing evolution. 
Evolutionary dynamics have been applied, for example, to develop self-driving cars \citep{abuzekry_comparative_2019}, design radio antennae for satellites \citep{oreilly_evolved_2005} , and advance the state of the art for machine learning \citep{vikhar_evolutionary_2016}. 

% Depending on space, we could potentially hammer a little more on the recent successes of evolutionary computation here

Across these various domains, experts are frequently interested in understanding not only how evolution has proceeded in the past, but where it might continue in the future. 
Such insights facilitate better understanding of the capabilities and limitations of their evolving system of interest. 
In these cases, experts are interested in understanding the search space of their particular problem and how evolution traverses that space (hereafter called ``evolutionary space"). 
Strong intuition for the topological properties of a given search space promote accurate inferences and predictions about how evolution will proceed in that space.

%As in all data analysis tasks, data visualization is a powerful tool for gaining such intuition.
%Good data visualizations are a particularly powerful way to build such intuition, as they do not need to rely on mathematical or domain-specific knowledge and can be broadly applicable across fields. 

To facilitate this intuition, search spaces are often represented as fitness landscapes.
In these representations of evolutionary space, high-fitness areas of the search space are represented as peaks and low-fitness areas are represented as valleys. 
This visual metaphor is ubiquitous and powerful, and its usefulness lies in its familiarity; it draws upon topographical representations which most people will have encountered in multiple other contexts. 
However, this familiarity can be misleading as it encourages us to think about landscapes in terms of only three spatial dimensions, whereas true evolutionary landscapes can have vastly more dimensions. 

Indeed, mathematical work suggests that the fitness landscape metaphor has mislead evolution researchers. 
High-dimensional fitness landscapes may have qualitatively different properties that lead evolution to behave very differently than it would in a low dimensional landscape \citep{agarwala_adaptive_2019}. 
This problem has lurked in the background of evolutionary theory research for over a decade \citep{kaplan_end_2008}. 
It is frequently stated as a caveat, but low-dimensional fitness landscapes continue to be used routinely for their immense intuition-building power.

Building intuitions with a data visualization known to potentially produce misleading intuitions is not ideal. 
To begin remedying this problem, we developed \textit{Tessevolve}, a web-enabled virtual reality (VR) based tool for visualizing fitness landscapes in 2D, 3D, and 4D. 
This platform enables users to begin building intuition for how fitness landscapes change as they gain dimensions. 
Additionally, it allows them to plot evolutionary data on top of the fitness landscape, to understand how the higher numbers of dimensions affect evolutionary dynamics. 
While four dimensions is still far fewer than most real world fitness landscapes contain, the ability to compare three dimensional landscapes to four dimensional landscapes enables a substantial advance in our ability to comprehend the effects of adding dimensions.

To our knowledge, Tessevolve is the first 4D visual representation of the evolution of a lineage and its surrounding landscape. 
Furthermore, for both 3D and 4D landscapes, domain experts found Tessevolve more intuitive and easier to use than existing visualization methods. 
Here we present the development and use cases for Tessevolve in detail, and highlight some of the insights made possible with this novel visual interface.