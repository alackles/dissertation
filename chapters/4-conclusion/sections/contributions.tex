\section{Contributions}

In summary, this dissertation makes the following contributions:

\begin{itemize}
    \item In \textbf{Chapter \ref{ch:re}}, I developed a rank-based metric for measuring genome connectivity that does not rely on underlying assumptions of baseline interaction. This model was able to correctly identify when loci were non-interactive in a case where existing metrics failed. Rank epistasis therefore allows us to better detect when epistasis is present, leading to more accurate understanding of interactivity within genomes. 
    \item In \textbf{Chapter \ref{ch:chm}}, we introduced the comparative hybrid method, an analytic tool for comparing computational cognitive structures. Using this method, we showed that connectivity and discretization are important components in the evolution of associative learning. These results demonstrate the value of a piecewise approach to analysis of evolving systems. 
    \item In \textbf{Chapter \ref{ch:vr-viz}}, I describe Tessevolve, a platform for visualizing and exploring fitness landscapes in up to four dimensions. Tessevolve expands our current ability to view and manipulate landscapes visually and allows us to view how patterns of ruggedness and mutational landscape connectivity change across dimensions. This platform represents a novel approach to the visualization of 3D and 4D fitness landscapes. 
\end{itemize}

As a whole, this dissertation contributes to our understanding of connected systems across multiple scales. it provides new tools and sets up a framework for thinking about connected systems in a digital evolution context, and lays groundwork for future studies on the computational study of evolvable networks and systems.

