\section{Future Directions}

I have so far provided overviews of three distinct approaches to evolutionary phenomena as well as representative examples of phenomena studied with each metric. These metrics dealt primarily with investigating the underlying connected structures of their respective evolvable systems. Here I present two branches of future directions to further investigate these systems: introducing information theory to current methods, and expanding our ability to interface with our data with novel metrics.

\subsection{Information-theoretic extensions to existing metrics}

The metrics introduced here have produced promising results for studying connectivity in evolvable systems. Further building upon these metrics would allow us to work towards a more complete picture of the underlying physical and mathematical constraints on our systems of interest. In particular, the quantitative and broadly applicable nature of information theory is a promising direction for metric expansion.


\subsubsection{Rank epistasis on information-theoretic networks}

The effect of epistasis on genotype structure can be quantified using information theory via \textit{statistical epistasis networks} \citep{moore_flexible_2006, mckinney_six_2012}.
In essence, epistatic loci or genes are nodes on the network, while edges represent interaction. 
These edges can be derived from the rank epistasis process by tracking which genotype rankings are altered for each double-mutant. 

Then, information theory as developed on networks can be applied to these epistatic networks; in particular, those information-theoretic metrics which were developed to quantify evolutionary networks \citep{sole_information_2004, adami_information_2011, carpi_analyzing_2011}. By viewing epistatic interactions as a network, we effectively get all the tools of network information theory to apply to epistasis.

\subsubsection{Adding informational network flow to the Comparative Hybrid Method}

In introducing the Comparative Hybrid Method, I discussed how the method gives a high-level, qualitative understanding of the impact of structure on function. 
However, quantifying this impact is also essential for deeper study of connectivity and constraint within computational architectures.
One direction for such quantification is analysis of informational network flow, or how information traverses the computational cognitive networks to create output.
Such a technique has been used in to investigate cognitive networks in the past and could readily be applied to comparative hybrid structures in the future \citep{bohm_information_2022}. 
By analyzing the informational network flow of hybrids versus their canonical counterparts, we could understand how changing structure on the phenotypic level changes information at the genotypic level, expanding our understanding of their mapping.

\subsection{Expanding our ability to interface with biological data}

In the past, our ability to interface with data has been limited to charts, tables, formulae, and figures--all of which are two-dimensional. However, biology moves and lives in three dimensions, and data about biology exists in many more dimensions than that. Therefore, leveraging emerging technology to expand our spatial and dimensional understanding of connections between system components will be essential to future breakthroughs in evolutionary theory.

\subsubsection{Refining intuition in Tessevolve}

While Tessevolve is a powerful prototype in its current form, feedback from colleagues and reviewers suggests several immediate improvements to its design. We are first interested in adding the ability to display only regions of fitness above a certain threshold value; this would allow users to scroll through the visualization in 4D and watch the fitness region grow or shrink to identify peaks and valleys. We are similarly interested in expanding the visualization of phylogenies through multiple dimensions through a similar fluid growing and shrinking of phylogeny space throughout dimensions.

\subsubsection{Creating audioscapes in VR}

The 3D visual landscape provided by virtual reality is a powerful tool for exploring fitness landscapes. 
An immediate next step in this research is to leverage the immersive sound capabilities of VR headsets to map landscape information to audio output. 
Such data sonification has been used in astrophysics to reveal patterns not obvious from mathematical or visual models alone \citep{gibney_how_2020}.
By leveraging audio as well as visual input, we can further expand the number of available dimensions in our audiovisual model.
Sonfiication would also expand the use case for these VR models to scientists who are blind/low-vision, colorblind, or otherwise may benefit from audio data interpretation over visual.

\subsubsection{Spatial models of genome structure}

Finally, in the next few years of my research I hope to merge the development of new frameworks with novel 3D data interfacing and develop multidimensional spatial models of genomes and genetic networks.
Genome models in artificial life and evolutionary computation are typically one-dimensional (i.e.~vectors), but biological genomes evolve and interact in three-dimensional space. 
Much like how protein structure dictates function, understanding the 3D form of genomes is essential to understanding complex processes such as gene regulation and modification \citep{mendizabal_epigenetics_2014, sotelo-silveira_entering_2018}.
Therefore, I would create novel models of genetic structures and networks to investigate how evolutionary forces are shaped by space and dimensionality.
Such modeling could range from comparing evolutionary dynamics of digital genomes of different dimensionality to creating entirely new evolving genome models within 3D physics emulators. Similar work has been undertaken in protein folding \cite{finkelstein_physics_2004}, but has not yet been extended to genome structure. 
These spatial models could hopefully provide insight into what it is like to be where science can only go in theory: inside the genome itself.

\section{Final Thoughts}

When we think about evolutionary biology we often think of finches and apes, birds and bees. But evolution in the abstract is more a process than it is a feature of any particular system or model, including those discussed and introduced here. Recognizing evolution as a process, and developing new approaches to study the intricacies of that process, gives us broad insight into the rich natural world in which we live. It is through this insight that we can begin to make connections out of the chaos.